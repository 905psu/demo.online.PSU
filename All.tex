%\documentclass[wert.tex]{subfiles}
%\documentclass[14pt,a4paper,twoside]{report}	% Размер основного шрифта и формата листа
\documentclass[12pt,a4paper,twoside]{report}
\usepackage{xltxtra,xunicode,polyglossia}						% Используется для вывода логотипа XeLaTeX
\newfontfamily\russianfont{Liberation Serif}
\usepackage{subfiles}
\setdefaultlanguage{russian}				% Основной язык текста
\setotherlanguage{english}					% Дополнительный язык текста
\linespread{1}							% Межстрочный интервал выбран полуторным
\usepackage[left=2.5cm,right=1.5cm,vmargin=2.5cm]{geometry} % Отступы по краям листа


\usepackage{xcolor,hyperref}
\definecolor{linkcolor}{HTML}{359B08} % цвет ссылок
\definecolor{urlcolor}{HTML}{799B03} % цвет гиперссылок
\hypersetup{pdfstartview=FitH,  linkcolor=linkcolor,urlcolor=urlcolor, colorlinks=true}

\usepackage{verbatim,indentfirst,cite,enumerate,float,amsmath,amssymb,amsthm,amsfonts}
\usepackage{graphicx,fontspec,subfigure,bm}
\newcounter{contnumeq}
\newcounter{contnumfig}
\newcounter{contnumtab}

\setcounter{contnumeq}{0}           % Нумерация формул: 0 --- пораздельно (во введении подряд, без номера раздела); 1 --- сквозная нумерация по всей диссертации
\setcounter{contnumfig}{0}          % Нумерация рисунков: 0 --- пораздельно (во введении подряд, без номера раздела); 1 --- сквозная нумерация по всей диссертации
\setcounter{contnumtab}{0}          % Нумерация таблиц: 0 --- пораздельно (во введении подряд, без номера раздела); 1 --- сквозная нумерация по всей диссертации

\graphicspath{
	{Chapter-1/Addition_of_angular_velocity/}
	{Chapter-1/Addition_of_motion/demo-1/}
%	{Chapter-1/Inertia/demo-1/}
%	{Chapter-1/Inertia/demo-2/}
%	{Chapter-1/Movement_along_the_loop/}
%	{Chapter-1/Movement_along_the_loop/}
%	{Chapter-1/Newton_laws_of_mechanics/demo-1/}
%	{Chapter-1/Newton_laws_of_mechanics/demo-2/}
%	{Chapter-1/The_law_of_momentum_conservation/demo-1/}
%	{Chapter-1/The_law_of_momentum_conservation/demo-2/}
%	{Chapter-2/Center_of_mass/demo-1/}
%	{Chapter-2/Center_of_mass/demo-2/}
%	{Chapter-2/Gyroscopic_effect/demo-1/}
%	{Chapter-2/Gyroscopic_effect/demo-2/}
%	{Chapter-2/Moment_of_inertia/demo-1/}
%	{Chapter-2/Moment_of_inertia/demo-2/}
%	{Chapter-2/Strange_roll/}
%	{Chapter-2/The_law_of_momentum_conservation/}
%	{Chapter-3/Ball_collision/}
%	{Chapter-3/Maxwell_pendulum/}
%	{Chapter-3/Potential_barrier/}
%	{Chapter-3/Rolling_cylinder/}
%	{Chapter-3/Transition_of_energy/}
%	{Chapter-4/Coriolis_force/}
%	{Chapter-4/Foucault_pendulum/}
%	{Chapter-4/Frictial_forces/demo-1/}
%	{Chapter-4/Frictial_forces/demo-2/}
%	{Chapter-4/Hooke_law/}
%	{Chapter-4/Inertial_forces/demo-1/}
%	{Chapter-4/Inertial_forces/demo-2/}
%	{Chapter-4/Inertial_forces/demo-3/}
			 }
\begin{document}
	\newpage 	% Создание новой страницы
	
	\setcounter{page}{2}% Нумерация начинается со второй страницы
	\renewcommand{\contentsname}{\center{Содержание}} 	% Использование Содержания вместо Оглавления
	\tableofcontents
	
	
	\subfile{Chapter-1/Addition_of_angular_velocity/Addition_of_angular_velocity.tex} % Сложение угловых скоростей
	\subfile{Chapter-1/Addition_of_motion/demo-1/Addition_speeds.tex} % Сложение движений
%	\subfile{Chapter-1/Inertia/demo-1/Pull_out_a_sheet.tex} % Выдергивание листа бумаги из-под стакана
%	\subfile{Chapter-1/Inertia/demo-2/Impulse_and_two_threads.tex} % Импульс силы. Гиря и две нити
%	\subfile{Chapter-1/Movement_along_the_loop/Loop.tex} % Петля Нестерова	
%	\subfile{Chapter-1/Newton_laws_of_mechanics/demo-1/Second_Law_two_carts.tex} % Второй закон Ньютона. Две тележки
%	\subfile{Chapter-1/Newton_laws_of_mechanics/demo-2/Third_Law_Archimedes.tex} % Третий закон Ньютона на примере силы Архимеда	
%	
%	\subfile{Chapter-2/Center_of_mass/demo-1/Center_of_mass.tex} % Отыскание центра масс
%	\subfile{Chapter-2/Center_of_mass/demo-2/Сone.tex} % Двойной конус
%	\subfile{Chapter-2/Gyroscopic_effect/demo-1/Gyroscope_on_a_rotating_platform.tex} % Гироскоп на вращающейся платформе
%	\subfile{Chapter-2/Gyroscopic_effect/demo-2/Top.tex} % Прецессия Гироскопа. Волчок
%	\subfile{Chapter-2/Moment_of_inertia/demo-1/Cross_Oberbeck_Pendulum.tex} % Крестообразный маятник Обербека
%	\subfile{Chapter-2/Moment_of_inertia/demo-2/Rotation_axis.tex} % Свободные оси вращения
%	\subfile{Chapter-2/Strange_roll/Strange_roll.tex} % Послушная и непослушная катушка
%	\subfile{Chapter-2/The_law_of_momentum_conservation/Zhukovsky_is_bench.tex} % Скамья Жуковского
%	
%	\subfile{Chapter-3/Ball_collision/Elastic_and_non-elastic_shock.tex} % Упругий и неупругий удар
%	\subfile{Chapter-3/Maxwell_pendulum/Maxwell_pendulum.tex} % Маятник Максвелла
%	\subfile{Chapter-3/Potential_barrier/Potential_barrier.tex} % Потенциальный барьер
%	\subfile{Chapter-3/Rolling_cylinder/Rolling_cylinder.tex} % Потенциальный барьер
%	\subfile{Chapter-3/Transition_of_energy/Transition_of_energy.tex} % Изгиб
%	
%	\subfile{Chapter-4/Coriolis_force/Coriolis_force.tex} % Сила Кориолиса
%	\subfile{Chapter-4/Foucault_pendulum/Foucault_pendulum.tex} % Маятник Фуко
%	\subfile{Chapter-4/Frictial_forces/demo-1/Dry_and_fluid_friction.tex} % Крестообразный маятник Обербека
%	\subfile{Chapter-4/Frictial_forces/demo-2/Dry_friction_force.tex} % Силы сухого трения
%	\subfile{Chapter-4/Hooke_law/Hooke_is_law.tex} % Закон Гука
%	\subfile{Chapter-4/Inertial_forces/demo-1/Plumbs_on_a_rotating_platform.tex} % Отвесы на вращающейся платформе
%	\subfile{Chapter-4/Inertial_forces/demo-2/Parabolic_surface_of_a_rotating_fluid.tex} % Параболическая поверхность вращающейся жидкости
%	\subfile{Chapter-4/Inertial_forces/demo-3/Dropping_a_chain_from_a_spinning_disk.tex} % Сбрасывание цепочки с вращающегося диска
\end{document}